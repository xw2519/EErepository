\documentclass[10pt,a4paper]{article}
\usepackage{graphicx}
\usepackage{biblatex}
\usepackage{parskip}
\usepackage{listings}
\usepackage{caption}
\usepackage{subcaption}
\usepackage{amsmath}
\usepackage{amssymb}
\usepackage{multicol}
\usepackage[most]{tcolorbox}

\lstset{basicstyle=\ttfamily, breaklines = true, tabsize=2}
\graphicspath{ {./Images/} }
\setlength{\parskip}{1em}
\begin{document}
%%%%%%%%%%%%%%%%%%%%%%%%%%%%%%%%%%%%%%%%%%%%%%%%%%%%%%%%%%%%%%%%%%%%%%%%%%%%%%%%%%%%%%%%%%%%%%%%%%%%%%%%%%

\begin{titlepage}
	\centering
	{\scshape\LARGE Imperial College London \par}
	\vspace{1cm}
    {\scshape\Large Discrete Mathematics: Year 2\par}
    \vspace{1.5cm}
	{\huge\bfseries Set Theory and Proof\par}
	\vspace{2cm}
	{\Large\ Xin Wang }
	\vfill
	{\large \today\par}
\end{titlepage}

%%%%%%%%%%%%%%%%%%%%%%%%%%%%%%%%%%%%%%%%%%%%%%%%%%%%%%%%%%%%%%%%%%%%%%%%%%%%%%%%%%%%%%%%%%%%%%%%%%%%%%%%%%

\begin{abstract}
    Set Theory is an important language and tool for reasoning. It is the basis of Mathematics and
    Computer Science since Logic is basically Set Theory. It is a useful tool for formalizing and
    reasoning about computations and its objects. \par 

    Proof is the activity of finding and discovering and confirming the truth. This chapter
    discusses how to write down proofs that are able to be communicated effectively. There is a
    balance to proofs between too much detail and too little (being too strict) - just like a sating
    in Italy: "It requires two people to make a good salad dress; a generous person to add oil and a
    mean person to add the vinegar". A tolerant openness and awareness is critical in
    discovering/understanding proofs while a strictness and discipline is required to write it down.
\end{abstract}

%%%%%%%%%%%%%%%%%%%%%%%%%%%%%%%%%%%%%%%%%%%%%%%%%%%%%%%%%%%%%%%%%%%%%%%%%%%%%%%%%%%%%%%%%%%%%%%%%%%%%%%%%%

\tableofcontents
\pagebreak

%%%%%%%%%%%%%%%%%%%%%%%%%%%%%%%%%%%%%%%%%%%%%%%%%%%%%%%%%%%%%%%%%%%%%%%%%%%%%%%%%%%%%%%%%%%%%%%%%%%%%%%%%%
\section{Mathematical argument}

Introduces basic mathematical notation and the various method of "argument". 

%%%%%%%%%%%%%%%%%%%%%%%%%%%%%%%%%%%%%%%%%%%%%%%%%%%%%%%%%%%%%%%%%%%%%%%%%%%%%%%%%%%%%%%%%%%%%%%%%%%%%%%%%%
\subsection{Logical notation}

For statements $A$ and $B$, there are common abbreviations:
\begin{enumerate}
    \item \textbf{Conjunction} of $A$ and $B$: $A$ and $B$ must both be TRUE or otherwise it is FALSE.
    $$
        A \textbf{ \& } B
    $$
    \item $A$ \textbf{implies} $B$: $B$ TRUE only if $A$ is TRUE.
    $$
        A \Rightarrow B
    $$
    \item $A$ \textbf{if} $B$: $A$ TRUE if and only if $B$ TRUE.
    $$
        A\Longleftrightarrow B
    $$
    \item not $A$: $A$ TRUE if $A$ is FALSE
    $$
        \neg A
    $$
\end{enumerate}

Common logic quantifiers:
\begin{enumerate}
    \item "There exists": $\exists$
    \item "For all": $\forall$
    \item "There exists $x$ such that $P(x)$":
    $$
        \exists x\:.\:P(x)
    $$
    \item "There exists some $x$ satisfying a property $P(x)$ but also that $x$ is the
    \textbf{unique} (only) object satisfying $P(x)$":
    $$
        (\exists x.P(x))\; \& \; (\forall y,z.P(y) \: \& \: P(z))\Rightarrow y = z
    $$
    Simplifies to:
    $$
        \exists !x.P(x)
    $$
\end{enumerate}

%%%%%%%%%%%%%%%%%%%%%%%%%%%%%%%%%%%%%%%%%%%%%%%%%%%%%%%%%%%%%%%%%%%%%%%%%%%%%%%%%%%%%%%%%%%%%%%%%%%%%%%%%%
\subsection{Patterns of proof}
%%%%%%%%%%%%%%%%%%%%%%%%%%%%%%%%%%%%%%%%%%%%%%%%%%%%%%%%%%%%%%%%%%%%%%%%%%%%%%%%%%%%%%%%%%%%%%%%%%%%%%%%%%
\subsubsection{Chains of implications}

To prove $A\Rightarrow B$, possible to show that assuming $A_1$ can be proved then possible to prove
$B$.
\begin{align*}
    A &\Rightarrow A_1 \\
    &\Rightarrow \dots \\
    &\Rightarrow A_n \\
    &\Rightarrow B
\end{align*}
Take care to differentiate between $\Rightarrow$ and $\Longleftrightarrow$ since sometimes the
reverse direction is untrue.

%%%%%%%%%%%%%%%%%%%%%%%%%%%%%%%%%%%%%%%%%%%%%%%%%%%%%%%%%%%%%%%%%%%%%%%%%%%%%%%%%%%%%%%%%%%%%%%%%%%%%%%%%%
\subsubsection{Proof by contradiction}

To show that $A$ is TRUE, sometimes it is only possible to show that $\neg A$ leads to a conclusion
which is FALSE - show $\neg A$ is not the case so $A$.

\textbf{Example 1}: Proof that $\sqrt{2}$ is irrational through proof by contradiction.
\begin{enumerate}
    \item Suppose $\sqrt{2}$ \textbf{is} rational - it can be written as ratio of two integers $p$
    and $q$:
    $$
        \sqrt{2} = \frac{p}{q}
    $$
    where assumed that $p$ and $q$ have no common factors - if there are then they are cancelled out

    Since $p$ and $q$ are simplified to the lowest terms, both $p$ and $q$ cannot be even - one
    must be odd.
    
    \item Square both sides:
    $$
        2 = \frac{p^2}{q^2}
    $$

    \item Implies that:
    $$
        p^2 = 2q^2
    $$

    $p^2$ is an even number since it is $2 \times q^2$. 

    \item Note $q$ is \textbf{also even} since if $q$ is odd then $q^2$ will be odd - odd number
    times odd number is always odd.
\end{enumerate}

%%%%%%%%%%%%%%%%%%%%%%%%%%%%%%%%%%%%%%%%%%%%%%%%%%%%%%%%%%%%%%%%%%%%%%%%%%%%%%%%%%%%%%%%%%%%%%%%%%%%%%%%%%
\subsubsection{Argument by cases}

Proof of $A_k \Rightarrow C$ requires the proof of $A_1\Rightarrow C$ to/and $A_k\Rightarrow C$.
Most often a proof breaks down into $k$ cases showing $A_1\Rightarrow C$,\dots,$A_k\Rightarrow C$.

\textbf{Example 1}: For all non-negative integers $a>b$, the difference of squares $a^2-b^2$ does
not give a remainder of $2$ when divided by $4$.
\begin{enumerate}
    \item Observe that:
    $$
        a^2-b^2 = (a+b)(a-b)
    $$
    \item Establish cases: 
    \begin{itemize}
        \item $a$ and $b$ are both even 
        \item $a$ and $b$ are both odd
        \item One is odd and the other one even.
    \end{itemize}
    \item Test the cases:
    \begin{itemize}
        \item Case 1: Both $a$ and $b$ are even: The product of two even numbers are divisible by 4,
        giving remainder 0.
        \item Case 2: Both $a$ and $b$ are odd: The product of two odd numbers are divisible by 4,
        giving remainder 0.
        \item Case 3: One is odd and the other even: Product is odd and remainder of 2 when divided
        by 4 - \textbf{a contradiction}.
    \end{itemize}
\end{enumerate} 


%%%%%%%%%%%%%%%%%%%%%%%%%%%%%%%%%%%%%%%%%%%%%%%%%%%%%%%%%%%%%%%%%%%%%%%%%%%%%%%%%%%%%%%%%%%%%%%%%%%%%%%%%%
\subsubsection{Existential properties}

Proving $\exists x\:.\:A(x)$, it suffices by showing an object $a$ such that $A(a)$. \par 

For example, one can show that $\exists x\:.\:A(x)$ by obtaining a contradiction - $\exists x \:.\:
\neg A(x)$

%%%%%%%%%%%%%%%%%%%%%%%%%%%%%%%%%%%%%%%%%%%%%%%%%%%%%%%%%%%%%%%%%%%%%%%%%%%%%%%%%%%%%%%%%%%%%%%%%%%%%%%%%%
\subsubsection{Universal properties}

Simplest way to prove $\forall x\:.\:A(x)$ is to let $x$ be an arbitrary element and show $A(x)$.
There are famous examples such as \textit{mathematical induction}.

%%%%%%%%%%%%%%%%%%%%%%%%%%%%%%%%%%%%%%%%%%%%%%%%%%%%%%%%%%%%%%%%%%%%%%%%%%%%%%%%%%%%%%%%%%%%%%%%%%%%%%%%%%
\subsection{Mathematical induction}


%%%%%%%%%%%%%%%%%%%%%%%%%%%%%%%%%%%%%%%%%%%%%%%%%%%%%%%%%%%%%%%%%%%%%%%%%%%%%%%%%%%%%%%%%%%%%%%%%%%%%%%%%%
\section{Sets}



%%%%%%%%%%%%%%%%%%%%%%%%%%%%%%%%%%%%%%%%%%%%%%%%%%%%%%%%%%%%%%%%%%%%%%%%%%%%%%%%%%%%%%%%%%%%%%%%%%%%%%%%%%
\end{document}